\section{Coordinate3D Class Reference}
\label{classCoordinate3D}\index{Coordinate3D@{Coordinate3D}}
{\tt \#include $<$coordinate3d.h$>$}

\subsection*{Public Methods}
\begin{CompactItemize}
\item 
{\bf Coordinate3D} ()
\item 
{\bf Coordinate3D} (double {\bf x}, double {\bf y}, double {\bf z})
\item 
{\bf $\sim$Coordinate3D} ()
\item 
void {\bf set\-X} (double {\bf x})
\item 
void {\bf set\-Y} (double {\bf y})
\item 
void {\bf set\-Z} (double {\bf z})
\item 
double {\bf get\-X} ()
\item 
double {\bf get\-Y} ()
\item 
double {\bf get\-Z} ()
\item 
double {\bf distance} (Coordinate3D $\ast$c)
\item 
{\bf Coordinate} $\ast$ {\bf projection} (double x\-Offset, double y\-Offset, double scale, double distance)
\item 
void {\bf apply\-Matrix} ({\bf Matrix}$<$ double $>$ $\ast$)
\item 
void {\bf translate} (Coordinate3D $\ast$)
\item 
void {\bf write} (std::ostream \&stream) const
\end{CompactItemize}
\subsection*{Protected Attributes}
\begin{CompactItemize}
\item 
double {\bf x}
\item 
double {\bf y}
\item 
double {\bf z}
\end{CompactItemize}


\subsection{Detailed Description}
This class handles coordinates. Remember that computers use coordinate systems that are upside-down. This means that the origin of the coordinate system lays in the top-left corner, so the Y axis is upside-down... You will notice that a lot of objects use pointers to Coordinates in constructors, etc. This is because the library needs to be generic, so you can use inherited Coordinates as Coordinates too (for example {\bf Spline\-Coordinate} {\rm (p.\,\pageref{classSplineCoordinate})} instances). \begin{Desc}
\item[Author: ]\par
Anthony Liekens \end{Desc}




\subsection{Constructor \& Destructor Documentation}
\index{Coordinate3D@{Coordinate3D}!Coordinate3D@{Coordinate3D}}
\index{Coordinate3D@{Coordinate3D}!Coordinate3D@{Coordinate3D}}
\subsubsection{\setlength{\rightskip}{0pt plus 5cm}Coordinate3D::Coordinate3D ()}\label{classCoordinate3D_a0}


Constructor. Constructs a coordinate. \index{Coordinate3D@{Coordinate3D}!Coordinate3D@{Coordinate3D}}
\index{Coordinate3D@{Coordinate3D}!Coordinate3D@{Coordinate3D}}
\subsubsection{\setlength{\rightskip}{0pt plus 5cm}Coordinate3D::Coordinate3D (double {\em x}, double {\em y}, double {\em z})}\label{classCoordinate3D_a1}


Constructor. Constructs a coordinate. \begin{Desc}
\item[Parameters: ]\par
\begin{description}
\item[{\em 
x}]Integer X-axis value of the coordinate (0 = leftmost point) \item[{\em 
y}]Integer Y-axis value of the coordinate (0 = top) \end{description}
\end{Desc}
\index{Coordinate3D@{Coordinate3D}!~Coordinate3D@{$\sim$Coordinate3D}}
\index{~Coordinate3D@{$\sim$Coordinate3D}!Coordinate3D@{Coordinate3D}}
\subsubsection{\setlength{\rightskip}{0pt plus 5cm}Coordinate3D::$\sim$Coordinate3D ()}\label{classCoordinate3D_a2}


Destructor. Destructs a coordinate. 

\subsection{Member Function Documentation}
\index{Coordinate3D@{Coordinate3D}!applyMatrix@{applyMatrix}}
\index{applyMatrix@{applyMatrix}!Coordinate3D@{Coordinate3D}}
\subsubsection{\setlength{\rightskip}{0pt plus 5cm}void Coordinate3D::apply\-Matrix ({\bf Matrix}$<$ double $>$ $\ast$)}\label{classCoordinate3D_a11}


\index{Coordinate3D@{Coordinate3D}!distance@{distance}}
\index{distance@{distance}!Coordinate3D@{Coordinate3D}}
\subsubsection{\setlength{\rightskip}{0pt plus 5cm}double Coordinate3D::distance (Coordinate3D $\ast$ {\em c})\hspace{0.3cm}{\tt  [inline]}}\label{classCoordinate3D_a9}


Returns the distance between this coordinate and another point. \begin{Desc}
\item[Parameters: ]\par
\begin{description}
\item[{\em 
c}]Coordinate3D of that other point \end{description}
\end{Desc}
\begin{Desc}
\item[Returns: ]\par
double \end{Desc}
\index{Coordinate3D@{Coordinate3D}!getX@{getX}}
\index{getX@{getX}!Coordinate3D@{Coordinate3D}}
\subsubsection{\setlength{\rightskip}{0pt plus 5cm}double Coordinate3D::get\-X ()\hspace{0.3cm}{\tt  [inline]}}\label{classCoordinate3D_a6}


Returns the X-axis value of the coordinate. \begin{Desc}
\item[Returns: ]\par
double \end{Desc}
\index{Coordinate3D@{Coordinate3D}!getY@{getY}}
\index{getY@{getY}!Coordinate3D@{Coordinate3D}}
\subsubsection{\setlength{\rightskip}{0pt plus 5cm}double Coordinate3D::get\-Y ()\hspace{0.3cm}{\tt  [inline]}}\label{classCoordinate3D_a7}


Returns the Y-axis value of the coordinate. \begin{Desc}
\item[Returns: ]\par
double \end{Desc}
\index{Coordinate3D@{Coordinate3D}!getZ@{getZ}}
\index{getZ@{getZ}!Coordinate3D@{Coordinate3D}}
\subsubsection{\setlength{\rightskip}{0pt plus 5cm}double Coordinate3D::get\-Z ()\hspace{0.3cm}{\tt  [inline]}}\label{classCoordinate3D_a8}


Returns the Y-axis value of the coordinate. \begin{Desc}
\item[Returns: ]\par
double \end{Desc}
\index{Coordinate3D@{Coordinate3D}!projection@{projection}}
\index{projection@{projection}!Coordinate3D@{Coordinate3D}}
\subsubsection{\setlength{\rightskip}{0pt plus 5cm}{\bf Coordinate} $\ast$ Coordinate3D::projection (double {\em x\-Offset}, double {\em y\-Offset}, double {\em scale}, double {\em distance})}\label{classCoordinate3D_a10}


\index{Coordinate3D@{Coordinate3D}!setX@{setX}}
\index{setX@{setX}!Coordinate3D@{Coordinate3D}}
\subsubsection{\setlength{\rightskip}{0pt plus 5cm}void Coordinate3D::set\-X (double {\em x})\hspace{0.3cm}{\tt  [inline]}}\label{classCoordinate3D_a3}


Set the X-axis value. \begin{Desc}
\item[Parameters: ]\par
\begin{description}
\item[{\em 
x}]double value \end{description}
\end{Desc}
\begin{Desc}
\item[Returns: ]\par
void \end{Desc}
\index{Coordinate3D@{Coordinate3D}!setY@{setY}}
\index{setY@{setY}!Coordinate3D@{Coordinate3D}}
\subsubsection{\setlength{\rightskip}{0pt plus 5cm}void Coordinate3D::set\-Y (double {\em y})\hspace{0.3cm}{\tt  [inline]}}\label{classCoordinate3D_a4}


Set the Y-axis value. \begin{Desc}
\item[Parameters: ]\par
\begin{description}
\item[{\em 
y}]double value \end{description}
\end{Desc}
\begin{Desc}
\item[Returns: ]\par
void \end{Desc}
\index{Coordinate3D@{Coordinate3D}!setZ@{setZ}}
\index{setZ@{setZ}!Coordinate3D@{Coordinate3D}}
\subsubsection{\setlength{\rightskip}{0pt plus 5cm}void Coordinate3D::set\-Z (double {\em z})\hspace{0.3cm}{\tt  [inline]}}\label{classCoordinate3D_a5}


Set the Z-axis value. \begin{Desc}
\item[Parameters: ]\par
\begin{description}
\item[{\em 
z}]double value \end{description}
\end{Desc}
\begin{Desc}
\item[Returns: ]\par
void \end{Desc}
\index{Coordinate3D@{Coordinate3D}!translate@{translate}}
\index{translate@{translate}!Coordinate3D@{Coordinate3D}}
\subsubsection{\setlength{\rightskip}{0pt plus 5cm}void Coordinate3D::translate (Coordinate3D $\ast$)}\label{classCoordinate3D_a12}


\index{Coordinate3D@{Coordinate3D}!write@{write}}
\index{write@{write}!Coordinate3D@{Coordinate3D}}
\subsubsection{\setlength{\rightskip}{0pt plus 5cm}void Coordinate3D::write (std::ostream \& {\em stream}) const}\label{classCoordinate3D_a13}


Write the coordinate object to a given outstream. \begin{Desc}
\item[Parameters: ]\par
\begin{description}
\item[{\em 
stream}]output stream \end{description}
\end{Desc}
\begin{Desc}
\item[Returns: ]\par
void \end{Desc}


\subsection{Member Data Documentation}
\index{Coordinate3D@{Coordinate3D}!x@{x}}
\index{x@{x}!Coordinate3D@{Coordinate3D}}
\subsubsection{\setlength{\rightskip}{0pt plus 5cm}double Coordinate3D::x\hspace{0.3cm}{\tt  [protected]}}\label{classCoordinate3D_n0}


\index{Coordinate3D@{Coordinate3D}!y@{y}}
\index{y@{y}!Coordinate3D@{Coordinate3D}}
\subsubsection{\setlength{\rightskip}{0pt plus 5cm}double Coordinate3D::y\hspace{0.3cm}{\tt  [protected]}}\label{classCoordinate3D_n1}


\index{Coordinate3D@{Coordinate3D}!z@{z}}
\index{z@{z}!Coordinate3D@{Coordinate3D}}
\subsubsection{\setlength{\rightskip}{0pt plus 5cm}double Coordinate3D::z\hspace{0.3cm}{\tt  [protected]}}\label{classCoordinate3D_n2}




The documentation for this class was generated from the following files:\begin{CompactItemize}
\item 
{\bf coordinate3d.h}\item 
{\bf coordinate3d.cpp}\end{CompactItemize}
